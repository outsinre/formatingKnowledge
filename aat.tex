\chapter{行测图行推理}

%https://www.zhihu.com/question/59668737

\section{规律总结}

首先分析图形\uline{外观},考察图形\uline{画法},得出图形里
的\uline{分部要素},也可称为\uline{元素}。

\begin{enumerate}
\item 边。三角形、四边形、五边形……
\item 曲线。
\item 圆。
\item 区。图形被分割成不同的封闭区域。
\item 交点。边上的交点。
\end{enumerate}

找出各图形元素后,再分析其\uline{规律}。

\begin{enumerate}
\item 行列。从行和列两个维度来分析规律。
\item 数量。每种元素的数量可能呈某种关系,如相等,差数列。还可
  以是行、列上数量关系。如边数,交点数,区域数等。
\item 笔画。是否是一笔画完。
\item 对称。如轴对称(竖轴、横轴),旋转对称(中心对称,180 度),
  不对称。如英文字母的对称性。还可能数对称轴(相同的、不同的)的数
  量。
\item 旋转。图形整体依次顺时针、逆时针旋转一个角度。
\item 翻转。图形整体以某轴翻转 180 度。如水平翻转。
\item 平移。某元素在图内平移一定距离。如小方格每次移顺时针移 3 个位
  置。
\item 相对。不同元素在图内相对位置发生变化。相对位置可以是内部、外
  部,相交、相接、分离,也可以是直角处、圆弧处,还可以是在最长边、
  最短边处。
\item 加减。相邻图形或元素之间加减组合出新图。如第 1 个图形和第 2
  个图形组合起来是第 3 个图形。常见的是九宫格里,行、列间图形是加
  减关系。有时甚至是先旋转再加减。
\item 类别。图形可以按其属性分类。如同为生活类用品,同属用腿、用手、
  手脚并用的运动项目。
\end{enumerate}

%%% Local Variables:
%%% mode: latex
%%% TeX-master: "main"
%%% End:
