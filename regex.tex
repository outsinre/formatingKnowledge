\chapter{Regular Expression}
\label{cha:regular-expression}

To learn regular expression, read
[Quickstart](https://www.regular-expressions.info/quickstart.html)
first. Check the bit-by-bit
\href{https://www.regular-expressions.info/tutorial.html}{authoritative
  tutorial} for details. To test and check regular expressions, go
to \href{https://regex101.com}{regex101.com}.

Pay attention to the terms \uline{character class} \lstinline|[ ]|
and \uline{capture group} \lstinline|( )|. A character class match
\textit{one and only one} character. So \lstinline|a[1-9]| does
not match a single character \verb|a|.

In regular expression, a dot matches any character except visual
newline. A negated character class matches a newline instead like
\lstinline|[^a-z]|.
To avoid that, use \lstinline|[^a-z\r\n]|.

Regex is not intended for or good at \textit{inverse} search. But
we can mimic this behavior by \textit{lookaround} at
\href{https://www.regular-expressions.info/lookaround.html}{Lookahead
  and Lookbehind Zero-Length Assertions}. It is quite
\href{https://stackoverflow.com/q/406230}{useful} when we want to
exclude something when matching.

For example, if the command is \lstinline|grep|, use option
\lstinline|-P| to enable \uline{PCRE} engine. The code below
excludes entries of which the URI part does not contain
\textit{itunes-assets}.

\begin{lstlisting}
cclog hpc access 201902141200 201902151200 | \
grep -P -m2 'http://aod\.itunes\.apple\.com/(?!itunes-assets).*headers='
\end{lstlisting}

However, we can resort to \lstinline|-v| to do the same work.

\begin{lstlisting}
cclog hpc access 201902141200 201902151200 | \
grep -m2 'http://aod\.itunes\.apple\.com/.*headers=' \
grep -v 'itunes-assets'
\end{lstlisting}

So when to use \textit{lookahead} and when to use
\textit{lookbehind}? From my experience, if you don't want
to match something \textit{immediately} following a
\textit{literal string}, then use \textit{lookahead}. Similarly,
if you don't want to match something immediately preceding a
literal literal string, use \textit{lookbehind}.

Like \lstinline|^| and
\lstinline|$| (for start and end of \textit{string}),
\lstinline|\b| is also an anchor for start and end of
\textit{word} like \lstinline|\bhello\b|. It can be called as
\textit{boundary}. The counterpart \lstinline|\B| matches at every
position where \lstinline|\b| cannot match.

To match a valid IPv4 address, use:

\begin{lstlisting}
^((25[0-5]|2[0-4][0-9]|[01]?[0-9][0-9]?)\.){3}(25[0-5]|2[0-4][0-9]|[01]?[0-9][0-9]?)$
\end{lstlisting}

When using regular expression, many commands
(i.e. \lstinline|sed|) use forward slash \verb|/| as
delimiters. In such cases, forward slash should be escaped. You
can use another delimter (i.e. \verb|#|) to avoid escaping.

Most of the time, forward slash is
\href{https://serverfault.com/q/892905}{not special} and escaping
is \href{https://stackoverflow.com/q/6076229}{not required}.

Recall in C language, a variable matches
\lstinline|[A-Za-z_][A-Za-z_0-9]*|.

%%% Local Variables:
%%% mode: latex
%%% TeX-master: "main"
%%% End:
