\chapter{汉语}
\label{cha:inputChinese}

对字体的引用要注意,如果一个字体族 family 有多个变种 \footnote{如
  regular, bold, italic 等。},请引用字体的 postscriptname.

默认情况下,\LaTeX{} 会把文档里用到的字体子集内嵌到生成的 PDF 里,
可以通过命令行 \textit{pdffonts filename.pdf} 查询,结果
如 \ref{cha:long-code-line} 所示。

此例说明,对应文档内嵌了七个字体子集。\LaTeX{} 文档通常只会用到字
体文件的部分 code point,所以只需要内嵌使用到的部分即可。每个字体
的前面都有一串无规则字母,代表对应字体文件被内嵌的子集。还有一列叫
sub 也说明是否是子集。

\section{CJK 包}

在英文文档里插入少量中文

\subsection{CJK 和 CJK* 环境}
在英文文档里用 CJK 包的 CJK 或 CJK* 环境插入少量汉字。后者会忽略汉
字后面的空格,推荐使用。

\subsection{字体}
bsmi~字体可能没有,尝试换一个如~gkai, ~gbsn~等。

\section{xeCJK 更好}
xeCJK 在处理细节上更好,关键是可方便设置中英文字体。用 xeCJK 则不需
要特殊环境包裹汉字,只需在``导言区''设置好中英文字体即可。

xeCJK 的 indentfirst 选项:
\begin{center}
  \verb|\usepackage[indentfirst]{xeCJK}|
\end{center}
已过时,推荐直接用 indentfirst 宏包:
\begin{center}
  \verb|\usepackage{indentfirst}|
\end{center}

\section{fontspec 和 ctex}
\label{sec:fontspec-ctex}

还有 \textit{fontspec} 和 \textit{ctex} 包可以实现中文输入,更多详
情请看 \href{https://www.zhstar.win/2015/02/05/LaTeX/}{\LaTeX{}
  post}.

\begin{itemize}
\item \verb|fontspec| sets western or Chinese fonts (i.e. main, sans, serif, bold, italic, and bolditalic).
\item \verb|CJK| sets Chinese, Japanese, and Korean fonts, punctuation etc. for TeX engine.
\item \verb|xeCJK| similar tof \verb|CJK| but for XeTeX engine.
\item \verb|CTeX| sets \verb|CJK/xeCJK| document layout.
\end{itemize}

\section{中英文空格}
我们想要的效果是英文字符前后自动加上空格。xeCJK 则会自动处理,但是
效果不是非常好。

CJK 会忽略 \TeX{} 源码里的空格。用 CJK 宏集里的 CJKspace 包,源码里
英文后空格会保留,中文后空格依然被 CJK* 压缩。还有一个方法是在需要空
格的地方用 tilde $\sim$.

\section{行首缩进}

中文习惯每段行首缩进两个汉字。英文有两种缩进格式。默认是每小节的第
一段缩进,后面的段不缩进,段间没有空行。另一种是不缩进,每段间留有
空行。

本文档只插入少量中文作例子,固保留英文默认缩进方式。如要换成中文缩
进,则用 indentfirst 包,使每节第一段也缩进,如下:
\begin{center}
  \begin{tabular}{l}
    \verb|\usepackage{indentfirst}| \\
    \verb|\setlength{\parindent}{2em}|
  \end{tabular}
\end{center}
如若换成英文第二种缩进,则
用\href{https://tex.stackexchange.com/a/40432}{parskip} 包,skip 表
示相临两段之间的间隙大小:
\begin{center}
  \begin{tabular}{l}
    \verb|\usepackage[parfill]{parskip}|
  \end{tabular}
\end{center}
也可用相对应的命令实现:
\begin{center}
  \begin{tabular}{l}
    \verb|\setlength{\parindent}{0pt}| \\
    \verb|\setlength{\parskip}{1ex plus 0.5ex minus 0.2ex}|
  \end{tabular}
\end{center}

\section{局部字体命令}

前面提到的方法都是全局生效,有没有类似 \verb|\emph{}| 和
\verb|\textit{}| 这种监时改中文字体的方法呢?

在 \textit{xecjk.tex} 中有如下代码:

\begin{lstlisting}[language=TeX,caption={New font family},label={lst:new-font-family}]
% set new font family
\setCJKfamilyfont{zhyahei}{msyh}[
  Path = fonts/,
  Extension = .ttf,
  BoldFont = {*bd}
]

% create font family command alias
\NewDocumentCommand{\yahei}{}{\CJKfamily{zhyahei}}
\end{lstlisting}

根据需求定义一个新的字体族 family 名 \textit{zhyahei},新字体族的
引用方法是:

\begin{lstlisting}[language=TeX,caption={Inline Chinese fonts},label={lst:inline-chn-fonts}]
{\CJKfamily{zhyahei}{这里输中文}}
% -or-
{\CJKfamily{zhyahei}这里输中文}
\end{lstlisting}

为了方便,我们还定义一个新 NewDocumentCommand 命令,便于快速引
用。\verb|{\yahei{这是新定义的雅黑字体样例。}}| 的效果
如:{\yahei{这是新定义的雅黑字体样例。}}

特别注总,命令要放在一个 group 里,即用 \verb|{}| 围起来,否则从当
前位置起,所有中文字体都改了。实际,fontspec 提供类似功能,具体请参
考 fontspec 文档。如:

\begin{center}
  \verb|\fontspec{⟨font name⟩}[⟨fontfeatures⟩]| 和 \verb|\newfontfamily|
\end{center}


%%% Local Variables:
%%% mode: latex
%%% TeX-master: "main"
%%% End:
