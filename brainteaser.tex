\chapter{Brain Teasers}
\label{cha:brain-teasers}

\subsubsection{补全数列空缺}

例 3 5 10 25 75 () 875. 观察此数列发现,相邻两项的数间没有什么特别关系,
相乘相加都不是。这时应考虑相邻三、四项项的关系,会发现 $3 \times 5 + 10
= 25$, 类似地有 $5 \times 10 + 25 = 75$. 因此空处的值应是 $10
\times 25 + 75 = 325$. 再来一个例子,5 6 16 28 60 (). 分析得出
$5 \ times 2 + 6 = 16$, 还有 $6 \times 2 + 16 = 28$. 所以空处的值
应是 $28 \times 2 + 60 = 116$. 此两例考虑连续多个数之间的递推关系,
类似于 Fibonacci Sequence.

\subsubsection{牛吃草}

一块匀均生长的草地,可供 27 头牛吃 6 周,或 23 牛吃 9 周,问多少牛
可吃 18 周?

我们先设一头牛吃草的速度是 a, 单位可以是 kg/w, 则 6 周吃 $27 \times 6
\cdot a$ kg 的草。设草的生长速度是 b, 单位可以是 kg/w, 那么 6 周
长了 $6 \cdot b$ kg 的草。那么是不是 $27 \times 6 \cdot a = 6b$ 呢?

不是!因为牛吃草前,草地上本来就有部分草,设为 c kg, 则 $27 \times
6 \cdot a = 6b + c$, 同理我们得出:

\[
  \begin{aligned}
    27 \times 6 \cdot a &= 6 \cdot b + c \\
    23 \times 9 \cdot a &= 9 \cdot b + c \\
    x \times 18 \cdot a &= 18 \cdot b + c \\
  \end{aligned}
\]

通过前两个方程,可得出 $b = 15 \cdot a$ 和 $c = 72 \cdot a$, 进而求出 $x = 19$.

\subsubsection{相向而行}

甲乙两地相距 90 米,A, B 二人分别从甲乙同时出发,在两地来回跑动。甲
的速度是 3 m/s,乙的是 2 m/s. 问 10 分钟内甲乙相遇几次?

第一次相遇时二人行使 90 米,时间是 $90 \div (3 + 2) = 18$s. 相遇后
二人背向而行,直至达到对端。此时二人又行使了 90 米,用时 18s.

二人再次相向而行,准备下一次相遇。此后一直重复此过程。可以发现相遇
一次二人行使了 180 米,花时 36s, 所以 10 分钟相遇次数是 $10 \times
60 \div 36 = 16\frac{2}{3}$, 结果是分数,那么是 16 次还是 17 次呢?

应是 17 次,因为第次相遇时,在 36s 的前半程,所以分数部分应四舍五
入!

上面这种方法不够明了。第一次相遇需 18s, 下一次相遇需 36s, 再下一次
相遇需 36s, 此后每一次相遇都需 36s. 我们找相遇的时间点更合理,得到
一个等差数列:

\[
  \begin{aligned}
    t_1 &= 18 \\
    t_2 &= 18 + 36 \\
    ... \\
    t_n &= 18 + (n - 1) \cdot 36
  \end{aligned}
\]

根据 $t_n = 10 \times 60$ 可以算出 $n = 17\frac{1}{6}$. 由此可知,
相遇 17 次,因为剩下的 1/6 还没到第 17 次相遇时间点。

\subsubsection{背向而行}

有一 300 米圆跑道,甲乙在同一地点背向而行,速度分别是 3.5m/s 和
4m/s. 问第 10 次相遇时,甲离出发地还有多远?

每相遇一次,两人共行走了 300 米,其中甲行走了 $\frac{3.5}{(3.5 +
  4)} \times 300 = 140$ 米,而乙行走了 160 米。

那么相遇 10 次时,甲累计走了 $10 \times 140 1400$ 米,而 $1400
\div 300 = 4\, \cdots \cdots\, 200$, 也就是说走了 4 圈后,第 5 圈走了
200 米,所以离出发点还剩 100 米。

\subsubsection{交换变量}

有两个整数 a = 5, b = 7, 不通过临时变量,如何交换二者的值?

方法一:异或 XOR. 异或的意思是,让两个数的二进制位互相比对,同为 0
或 1 时结果为 0, 有一个 1 一个 0 (相异)时结果为 1. 所以任何数异
或自己的结果是 0. 对应的 C 语言结果是:

\begin{lstlisting}[ language=C,caption={Swap by XOR} ]
a = a^b;
b = a^b;
a = a^b;
\end{lstlisting}

方法二:加减法。

\begin{lstlisting}[ language=C,caption={Swap by Arithmetic Operations} ]
a = a + b;
b = a - b;
a = a - b;
// Or
a = (a + b) - (b = a)
\end{lstlisting}

%%% Local Variables:
%%% mode: latex
%%% TeX-master: "main"
%%% End:
