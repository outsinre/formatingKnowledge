\chapter{TCP/IP}
\label{cha:tcpip}

\section{Segment Datagram and Frame}
\label{sec:segment-datagram-frame}

Before everything else, let's have a look at terminologies
\textit{segment}, \textit{datagram} and \textit{frame}.

Generally speaking, they refer to a \textit{unit} of data
transmitted through the protocol stack without differentiating
protocols.

However, when talking about \textit{segment}, we mean a TCP
packet, including both the TCP headers and TCP payload. On the
other hand, \textit{datagram} means payload and headers of an IP
packet. The two terms are also called \textit{packet} in
general. Interestingly, a UDP packet is also called
datagram. Maybe this is due to UDP's \textit{at best} strategy.

Lastly, a \textit{frame} is a \textit{Link layer} (mostly the
Ethernet) unit. But it is weird to call it a packet.

You may also hear of \textit{fragment}. When an IP datagram is too
large to fit a MTU \ref{sec:mss-mtu}, the payload part is split
into smaller parts, forming multiple new units. This is also
called \textit{fragmentation}.

By the way, in the Physical layer, we usually use
\textit{transmit} or \textit{transmission}. More details about
those terms, read
\href{https://stackoverflow.com/a/11637061}{Definition of Network
  Units}.

\section{MTU and MSS}
\label{sec:mtu-mss}

Let's talk about \textit{Maximum Transmission Unit (MTU)}
first. It is the largest IP packet that can be transmitted over
\textbf{all} links from source to destination. So MTU refers to
the maximum payload of Ethernet frames, \textbf{not} including
frame headers (20 bytes).

It is a physical parameter (i.e. set in router) that specifies the
size of the largest protocol data unit (i.e. IP datagram) that can
be sent in a single link layer transmission. MTU usually appear in
association with network Interfaces (i.e. routers, PC NICs, serial
ports etc.).

\textit{Maximum Segment Size (MSS} is confined to MTU. It refers
to the maximum amount of application-layer data that can be placed
in a TCP segment, \textbf{not} including segment headers or Option
fields. So we can say it is a parameter of the \textit{Options
  fields}. Although it is named after \textit{segment}, MSS does
not count TCP headers and Options field.

So both terms refer to payload only! Their relations can be
expressed by equation \eqref{eq:mtu-mss}.

\begin{equation}
  \label{eq:mtu-mss}
    \text{MTU} = \text{IP Headers (20 bytes)} + \text{TCP Headers (20 bytes)} \text{ [+ TCP Options]} + \text{MSS}
\end{equation}

For general Ethernet links, MSS is $1500 - 20 - 20 = 1460$. For
Point-to-Point Protocol over Ethernet (PPPoE), it requires extra 8
bytes (6 bytes PPPoE plus 2 bytes PPP) overhead. Therefore the MTU
is $1500 - 8 = 1492$ bytes and corresponding MSS is 1452 bytes.

% https://blog.zenlab.it/traceroute-vs-ping-vs-mtr/

%%% Local Variables:
%%% mode: latex
%%% TeX-master: "main"
%%% End:
