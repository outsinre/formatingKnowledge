\begin{appendices}

\chapter{Too Big to Fit}
\label{cha:too-big-fit}

\section{Appendix Tips}
\label{cha:appendix-tips}

The \textit{appendix} package provides \textit{appendices}
environment that renames each chapter to Appendix A, Appendix B,
etc. Similarly, sections are renamed to A.1, A.2,
etc. respectively.

\section{Embedded fonts in PDF}
\label{sec:pdffonts}

Here is an example of PDF file generated by \LaTeX{}. We use
command tool \textit{pdffonts} to examine embedded fonts:

\begin{lstlisting}[language={},caption={\LaTeX{} 内嵌字体},label={pdffonts},frame={tb},basicstyle=\tiny\ttfamily,linewidth=.88\textwidth]
name                                 type              encoding         emb sub uni object ID
------------------------------------ ----------------- ---------------- --- --- --- ---------
TVICFK+Tinos                         CID TrueType      Identity-H       yes yes yes      5  0
XYTJRZ+AdobeSongStd-Light-Identity-H CID Type 0C       Identity-H       yes yes no       7  0
NJETOY+Tinos-Italic                  CID TrueType      Identity-H       yes yes yes      9  0
VMQFJA+Tinos-Bold                    CID TrueType      Identity-H       yes yes yes     15  0
ILVYSG+NotoSansHans-Bold-Identity-H  CID Type 0C       Identity-H       yes yes yes     20  0
HEDEEX+migu-1m-regular               CID TrueType      Identity-H       yes yes yes     50  0
KQTNVZ+CMSY10                        Type 1C           Builtin          yes yes no      55  0
\end{lstlisting}  

\section{pgfplotstable template}
\label{sec:pgfplotstable-template}

\begin{minipage}[tbp]{1.0\linewidth}
\begin{lstlisting}[language=TeX,caption={pgfplotstable
template},label={lst:pgfplotstable-template},basicstyle=\scriptsize\ttfamily,linewidth=\textwidth]
\begin{table}[tbp]
  \centering{} \pgfplotstabletypeset[ multicolumn names, % allows
  to have column header name col sep=comma, % the seperator in our
  .csv file
  display columns/0/.style={ % numbering starts at 0
    column name=$Ampere$, % header name of first column
    column type={S},string type % use siunitx for formatting
  }, display columns/1/.style={ column name=$Voltage$, column
    type={S},string type }, display columns/2/.style={ column
    name=$Energy$, column type={S},string type }, every head
  row/.style={ before row={\toprule}, % have a rule at top
    after row={ \si{\ampere} & \si{\volt} & \si{joule} \\ % the
      siunitx units seperated by & \midrule % rule under units
    } }, every last row/.style={ after row=\bottomrule % rule at
    bottom }, ]{pgfplotstable.csv} % filename/path to file
  \caption{Table automation from .csv file.}
  \label{table-automation-from-csv}
\end{table}
\end{lstlisting}    
\end{minipage}

\begin{minipage}[tbp]{1.0\linewidth}
\begin{lstlisting}[language=TeX,caption={pgfplots
template},label={lst:pgfplots-template},basicstyle=\tiny\ttfamily,linewidth=\textwidth]
\begin{figure}[!h]
  \centering
  \begin{tikzpicture}
    \begin{axis}[
      width = \linewidth, % Scale the plot to \linewidth
      grid = major, grid style = dashed,
      xlabel = Voltage $U$, ylabel = Currency $I$, % Set the labels
      x unit = \si{\volt}, y unit = \si{\ampere}, % Set the respective units
      % axis lines = left % only display the left and bottom axes
      legend style = { at = {(0.5,-0.2)}, anchor = north }, % Put
      the legend below the plot x tick label style = { rotate =
        90, anchor = east } %
      Display labels sideways ]
      % add a plot from table; you select the columns by using the
      % actual column header name in the .csv file
      \addplot table[x=value 1,y=value 2,col
      sep=comma]{pgfplots.csv}; \legend{$U$ - $I$}
      % add another plot
      \addplot {x^2 - 2*x + 1}; % add a tailing semicolon
      \addlegendentry{$x^2 - 2x + 1$} % use addlegendentry instead
      of legend
    \end{axis}
  \end{tikzpicture}
  \caption{pgfplots by table csv file}
  \label{fig:pgfplots-by-table-csv-file}
\end{figure}
\end{lstlisting}
\end{minipage}

\section{missTime}
\label{sec:bash-misstime}

\begin{minipage}{0.8\linewidth}
\begin{lstlisting}[language=bash,caption={missTime Line by Line},label={py-misstime-line-by-line},basicstyle=\tiny\ttfamily,linewidth=\textwidth]
#!/bin/bash

if [[ "$1" = @(-h|--help) ]]
then
    printf 'Usage: bash missTime [<refr-Yy/Nn> [<number-of-test> [<input-file>]]]\n
refr-Yy/Nn:\t\trefresh or not (default: Y).
number-of-test:\t\tnumber of tests (default: 5).
input-file:\t\tfile contains the log.\n
bash missTime y 2 input.raw
cclog hpc access 5m | grep -E "TCP_HIT.*iosapps.itunes.apple.com.*headers=" | tail -2 | bash missTime\n'

    exit 0
fi

_refr_opt=${1:-Y}
if [[ ! "$_refr_opt" =~ ^[YyNn]$ ]]
then
    printf '%s: Yy/Nn expected!\n' "$1"
    exit 1
fi

_num_of_tests=${2:-5}
if [[ ! "$_num_of_tests" =~ ^[0-9]*$ ]]
then
    printf '%s: integer expected!\n' "$_num_of_tests"
fi

_infd=0; [[ -f "$3" ]] && exec {_infd}< "$3"

read -r _ _ _ _ip _ < <(ip -4 -o addr show scope global dev bond0) ; _ip="${_ip%/*}"

_headers_regex=$'\^\~\$headers=\'(.*)\)@\|#\(\'' #'
_https_regex='^https://'
_refr_port="770"

_log_dir="${HOME}/logs-missTime"
mkdir -p "${_log_dir}"
for f in *.log; do mv "$f" "${_log_dir}/" 2>/dev/null; done

while IFS=$' \t\n' read -p "Log expected:" -u "${_infd}" -r line
do
    mapfile -t _log_array < <( awk -F'[[:space:]]*\\)@(in|out)#\\([[:space:]]*' '{ for (i = 1; i <= NF; ++i) print $i; }' <<< "${line}" )
    
    _std_log="${_log_array[0]}" _inr_log="${_log_array[1]}"
    [[ "${_log_array[2]}" ]] && _ext_log="${_log_array[2]}"

    read -ra _std_array <<< "${_std_log}" ; _log_time="${_std_array[0]}" _url="${_std_array[6]}"

    if [[ ! "${_inr_log}" =~ $_headers_regex ]]
    then
	printf '%s: log format error.\n' "${_log_time}"
	continue
    fi
    tmp_headers="${BASH_REMATCH[1]})@|#(" _headers_array=()
    while [[ $tmp_headers ]]
    do
	_headers_array+=( "${tmp_headers%%\)\@\|\#\(*}" )
	tmp_headers="${tmp_headers#*\)\@\|\#\(}"
    done; unset tmp_headers
    
    IFS='/:' read -r _scheme _ _ _host _uri <<< "${_url}"
    if [[ "${_scheme}" == "https" ]]
    then
	_protocol_port=443
    else
	_protocol_port=80
    fi
    _resolve="${_host}:${_protocol_port}:${_ip}"

    for f in full real; do declare "_$f"="${_log_time}-${_num_of_tests}-$f.log"; done
    for f in "${_full}" "${_real}"; do echo -n >| "$f"; done  
    {
	printf -- '########## missTime ##########\n'
	printf 'HN:\t%s\nIP:\t%s\nLOG:\t%s\nRefr:\t%s\nTest:\t%s\n\n' "$(hostname)" "${_ip}" "${_log_time}" "${_refr_opt}" "${_num_of_tests}"
	printf 'cclog:\n%s\n\n' "${line}"
	printf 'standard part:\n%s\n\ninternal part:\n%s\n\n' "${_std_log}" "${_inr_log}"
	[[ "${_ext_log}" ]] && printf 'external part:\n%s\n\n' "${_ext_log}"
	printf 'curl -ksSI \\\n' ; printf -- $'-H \'%s\' \\\n' "${_headers_array[@]}" ; printf -- $'-H \'x-c3-debug:enabled\' \\\n--resolve \'%s\' \\\n\'%s\'\n\n' "${_resolve}" "${_url}" #'
	printf 'curl -ksSI ' ; printf -- '-H %q ' "${_headers_array[@]}" ; printf -- '-H %q --resolve %q %q\n\n' "x-c3-debug:enabled" "${_resolve}" "${_url}"
    } | tee -a "${_full}"
    
    _headers_array=( "${_headers_array[@]/#/-H}" )
    printf '' >| "curl-reply.log"
    ( set -x; { curl -ksSI "${_headers_array[@]}" -H "x-c3-debug:enabled" --resolve "${_resolve}" "${_url}" ; } 2>&1 ) | tee -a "${_full}" "curl-reply.log" 

\end{lstlisting}
\end{minipage}

\begin{minipage}{0.8\linewidth}
\begin{lstlisting}[language=bash,caption={missTime Line by Line},label={py-misstime-line-by-line},basicstyle=\tiny\ttfamily,linewidth=\textwidth]
    [[ "$_refr_opt" == [Nn] ]] && continue
    
    declare -A _reply_array=()
    while IFS=' :' read -r key value
    do
	[[ "$key" && "$key" != [[:space:]] && "$key" != + ]] || continue
	_reply_array[${key}]=${value%$'\r'}
    done < "curl-reply.log"
    
    {
	# for key in "${!_reply_array[@]}"; do printf '%s: %s\n' "$key" "${_reply_array[$key]}" ; done ; printf '\n'
	printf 'X-Cache: %s\n' "${_reply_array[X-Cache]}" ; printf 'X-Cache-Remote: %s\n' "${_reply_array[X-Cache-Remote]}" ; printf 'CC_CACHE: %s\n\n' "${_reply_array[CC_CACHE]}"
    } | tee -a "${_full}"

    shopt -q extglob; _extglob_set=$? ; (( _extglob_set )) && shopt -s extglob
    _delete_url=${_reply_array[X-True-Cache-Key]#http?(s)://}
    (( _extglob_set )) && shopt -u extglob

    [[ "${_scheme}" == "https" ]] && _delete_url=${_delete_url/\//:443/}
    _refr_url=http://127.0.0.1:${_refr_port}/delete/${_delete_url}
    printf 'refr url: %s\n' "${_refr_url}" | tee -a "${_full}"
    
    for i in $(seq 1 ${_num_of_tests})
    do
	printf -- '---%sth---\n' "${i}" | tee -a "${_full}" "${_real}"

	printf 'refr code: ' | tee -a "${_full}"
	curl -ksSo /dev/null -w "%{response_code}" -H "User-Agent: DataDelete" "${_refr_url}" | tee -a "${_full}"
	printf '\n' | tee -a "${_full}"
	
	sleep 0.05s
	{ time curl -kLsSvo /dev/null "${_headers_array[@]}" -H "x-c3-debug:enabled" --resolve "${_resolve}" "${_url}" 2>&1 | awk 'BEGIN {IGNORECASE=1}; /<[[:space:]].*[-_]cache:/ {print substr($0, 3) }' ; } 2>&1 | tee -a "${_full}"

	ed -s "${_full}" <<< $'$-3d\nw'
	ed -s "${_full}" <<< $'$-3,-2p' >> "${_real}"
    done
    printf '%s\n' '1d' w | ed -s "$_full" # sed -i -e '1d' "$_full"
    
done

rm -f "curl-reply.log"
\end{lstlisting}
\end{minipage}









\newpage{}
\listoffigures{}
\listoftables{}
\lstlistoflistings{}

\end{appendices}

%%% Local Variables:
%%% mode: latex
%%% TeX-master: "main"
%%% End:
