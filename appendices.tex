\begin{appendices}

\chapter{Long Code Line}
\label{cha:long-code-line}

\section{Appendix Notes}
\label{cha:appendix-notes}

The \textit{appendix} package provides \textit{appendices}
environment that renames each chapter to Appendix A, Appendix B,
etc. Similarly, sections are renamed to A.1, A.2,
etc. respectively.

\section{Embedded fonts in PDF}
\label{sec:pdffonts}

Here is an example of PDF file generated by \LaTeX{}. We use
command tool \textit{pdffonts} to examine embedded fonts:

\begin{lstlisting}[language={},caption={\LaTeX{} 内嵌字体},label={pdffonts},frame={tb},basicstyle=\tiny\ttfamily,linewidth=.88\textwidth]
name                                 type              encoding         emb sub uni object ID
------------------------------------ ----------------- ---------------- --- --- --- ---------
TVICFK+Tinos                         CID TrueType      Identity-H       yes yes yes      5  0
XYTJRZ+AdobeSongStd-Light-Identity-H CID Type 0C       Identity-H       yes yes no       7  0
NJETOY+Tinos-Italic                  CID TrueType      Identity-H       yes yes yes      9  0
VMQFJA+Tinos-Bold                    CID TrueType      Identity-H       yes yes yes     15  0
ILVYSG+NotoSansHans-Bold-Identity-H  CID Type 0C       Identity-H       yes yes yes     20  0
HEDEEX+migu-1m-regular               CID TrueType      Identity-H       yes yes yes     50  0
KQTNVZ+CMSY10                        Type 1C           Builtin          yes yes no      55  0
\end{lstlisting}  

\section{pgfplotstable template}
\label{sec:pgfplotstable-template}

\begin{minipage}[tbp]{1.0\linewidth}
\begin{lstlisting}[language=TeX,caption={pgfplotstable
template},label={lst:pgfplotstable-template},basicstyle=\scriptsize\ttfamily,linewidth=\textwidth]
\begin{table}[tbp]
  \centering{} \pgfplotstabletypeset[ multicolumn names, % allows
  to have column header name col sep=comma, % the seperator in our
  .csv file
  display columns/0/.style={ % numbering starts at 0
    column name=$Ampere$, % header name of first column
    column type={S},string type % use siunitx for formatting
  }, display columns/1/.style={ column name=$Voltage$, column
    type={S},string type }, display columns/2/.style={ column
    name=$Energy$, column type={S},string type }, every head
  row/.style={ before row={\toprule}, % have a rule at top
    after row={ \si{\ampere} & \si{\volt} & \si{joule} \\ % the
      siunitx units seperated by & \midrule % rule under units
    } }, every last row/.style={ after row=\bottomrule % rule at
    bottom }, ]{pgfplotstable.csv} % filename/path to file
  \caption{Table automation from .csv file.}
  \label{table-automation-from-csv}
\end{table}
\end{lstlisting}    
\end{minipage}

\begin{minipage}[tbp]{1.0\linewidth}
\begin{lstlisting}[language=TeX,caption={pgfplots
template},label={lst:pgfplots-template},basicstyle=\tiny\ttfamily,linewidth=\textwidth]
\begin{figure}[!h]
  \centering
  \begin{tikzpicture}
    \begin{axis}[
      width = \linewidth, % Scale the plot to \linewidth
      grid = major, grid style = dashed,
      xlabel = Voltage $U$, ylabel = Currency $I$, % Set the labels
      x unit = \si{\volt}, y unit = \si{\ampere}, % Set the respective units
      % axis lines = left % only display the left and bottom axes
      legend style = { at = {(0.5,-0.2)}, anchor = north }, % Put
      the legend below the plot x tick label style = { rotate =
        90, anchor = east } %
      Display labels sideways ]
      % add a plot from table; you select the columns by using the
      % actual column header name in the .csv file
      \addplot table[x=value 1,y=value 2,col
      sep=comma]{pgfplots.csv}; \legend{$U$ - $I$}
      % add another plot
      \addplot {x^2 - 2*x + 1}; % add a tailing semicolon
      \addlegendentry{$x^2 - 2x + 1$} % use addlegendentry instead
      of legend
    \end{axis}
  \end{tikzpicture}
  \caption{pgfplots by table csv file}
  \label{fig:pgfplots-by-table-csv-file}
\end{figure}
\end{lstlisting}
\end{minipage}

\newpage{}
\listoffigures{}
\listoftables{}
\lstlistoflistings{}

\end{appendices}

%%% Local Variables:
%%% mode: latex
%%% TeX-master: "main"
%%% End:
