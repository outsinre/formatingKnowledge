\chapter{ICT Operation}
\label{cha:ict-operation}

\lstset{language=bash}

\section{Cheat Sheet}
\label{sec:linux-cheatsheet}

Table \ref{tab:verify-linux-release} for release information;
\ref{tab:linux-account-management} for account information; table
\ref{tab:cmd-rock-tips} for handy tips.

\begin{table}[!htb]
  \centering
  \begin{tabular}[!htb]{l|l}
    \toprule
    Command & Usage \\
    \midrule
    \lstinline|fgrep -qa docker /proc/1/cgroup; echo $?| & Docker container \\
    \lstinline|cat /etc/{os,redhat}-release; rpm -q centos-release| & Distribution name \\
    \bottomrule
  \end{tabular}
  \caption{Verify Linux Release}
  \label{tab:verify-linux-release}
\end{table}

\begin{table}[!htb]
  \small
  \centering
  \begin{tabular}[!htb]{l|l}
    \toprule
    Command & Usage \\
    \midrule
    \lstinline|groupadd -g 2000 name| & Creata a group \\
    \lstinline|useradd -ms /bin/bash -u 1000 -g name name| & Create an account and specify the login group \\
    \lstinline|groupmod -g 1000 -n newname name| & Change group ID and name \\
    \lstinline|usermod -u 1000 -g newname -aG wheel name| & Change account ID, login group ID, and append to the 'wheel' group \\
    \lstinline/gpasswd [-a | -d] name name/ & Add or delete account 'name' to group 'name' \\
    \lstinline/getent [ passwd | group | hosts ]/ & Administrative database \\
    \lstinline|groups username| & Groups username joins \\
    \lstinline|users| & Users logged in to the system \\
    \lstinline|id| & Print uid, gid and groups \\
    \bottomrule
  \end{tabular}
  \caption{Linux Account Management}
  \label{tab:linux-account-management}
\end{table}

\begin{table}[!htb]
  \small
  \centering
  \begin{tabular}[!htb]{l|l}
    \toprule
    Command & Usage \\
    \midrule
    \lstinline|cat >> file.md << EOF| & Append contents to a file from STDIN with a \textit{here document} \\
    \bottomrule
  \end{tabular}
  \caption{Rock Tips}
  \label{tab:cmd-rock-tips}
\end{table}

\section{Linux Tools}
\label{sec:linux-tools}

\subsection{free}
\label{sec:linux-free}

\lstinline|free -hwt| commands display the amount of memory
allocations, by analyzing the contents of file
\lstinline|/proc/meminfo|:

\begin{lstlisting}[caption={Free Memory},label={lst:free-mem},basicstyle=\tiny\ttfamily]
              total        used        free      shared     buffers       cache   available
Mem:           3.7G        781M        1.5G        160M        558M        985M        2.8G
Swap:          1.0G          0B        1.0G
Total:         4.7G        781M        2.5
\end{lstlisting}

Always keep euqation:

$$\text{total} = \text{used} + \text{free} + \text{buffers} + \text{cache}$$

in mind.

\begin{itemize}
\item \textit{used} memory is allocated to user space processes.
\item \textit{free} memory is not allocated or freed up.
\item textit{shared} memory (mostly by \textit{tmpfs} is part of
  \textit{used}, shared among processes.
\item Kernel \textit{buffers} and \textit{cache} is allocated to
  kernel itself. \textit{cache} refers to memory used by the
  \textit{page cache} and \textit{reclaimable slab}s. For details
  about \textit{cache} and \textit{slab}, please read
  \ref{sec:os-memory}.
\item \textit{available} is the estimated amount of memory for
  starting new applications, \textit{without} swapping.
\end{itemize}

Ideally, \textit{available} should be the sum of \textit{free},
\textit{buffers} and \textit{cache}. However, \textbf{not all}
\textit{reclaimable slabs} can be freed in time as some may be in
use. So the actual \textit{available} size is slightly smaller.

%%% Local Variables:
%%% mode: latex
%%% TeX-master: "main"
%%% End:
